\section{Introduction}
\label{sec:intro}


%
% About AgentSpeak(L)
%
{\asl} has been introduced in \cite{Rao96}  with the purpose of defining an expressive, 
abstract language capturing the main aspects of the Belief-Desire-Intention architecture~\cite{Bratman88,Georgeff:1987:RRP:1863766.1863818}, featuring a formally defined semantics and an abstract interpreter.
%
The starting point to define the language were real-world implemented systems, namely the
Procedural Reasoning System (PRS)~\cite{Ingrand:1992:ARR:629535.629890} and the Distributed Multi-Agent Reasoning System (dMARS).
%

%
% Concrete APL
% 
{\asl}  and PRS have become a main reference for implementing concrete Agent Programming Languages 
based on the BDI model: 
%
main examples are Jason~\cite{jason06,bordini:07} and ASTRA~\cite{DBLP:conf/prima/CollierRL15}.
%
Besides Agent Programming Languages, the {\asl} model has been adopted
as the main reference to development several BDI agent-based
frameworks and technologies~\cite{BordiniMAPlpa,BordiniMAPlta} as well
as serving as inspiration for theoretical work aiming to formalise
aspects of BDI agents and agent programming
languages~\cite{DBLP:conf/kr/WinikoffPHT02,DBLP:conf/promas/DennisFBFW07,DBLP:journals/aamas/BordiniFVW06}.


%
% Contribution
%
Existing Agent Programming Languages extended the language with
constructs and mechanisms making it practical from a programming point
of view~\cite{jason06}.
%
Besides, proposals in literature extended the model is order to make
it effective for specific kinds of systems --- e.g., real-time
systems~\cite{Vikhorev:2011:APP:2030470.2030529} -- or to
improve the structure of programs, e.g. in terms of
modularity~\cite{Madden2010,Nunes2014}.
%%RHB: Not sure what should go in XXX above?

%
Along this line, in this paper we describe a novel extension of the
{\asl} model --- called {\aser} --- featuring \emph{plan
  encapsulation}, i.e. the possibility to define plans that fully
encapsulate the strategy to achieve the corresponding goals,
integrating both the pro-active and the reactive behaviour.
%
% Key points
%
This extension turns out to bring a number of important benefits to
agent programming based on the BDI model, namely:
%
\begin{itemize}
\item improving the overall readability of the agent source code,
  reducing fragmentation and increasing modularity;
\item promoting a more goal-oriented programming style, enforcing yet
  preserving the possibility to specify purely reactive behaviour,
  properly encapsulated into plans for goals;
\item improving intention management, enforcing a one-to-one relation
  between intentions and goals --- so every intention is related to a
  (top-level) goal;
\item improving failure handling, in particular making it easier the
  management of failures related to plans for reacting to environment
  events;
\end{itemize}
%
% \noindent Besides the benefits in terms of agent programming, the approach reduces the gap between the design level and the programming level,  ...
% \item facilitate goal-based reasoning -- ...
% \end{itemize}

\noindent The remainder of the paper is organised as follows:
%
first we describe in details the motivations that lead to the proposal
of a new AgentSpeak extension (Section~\ref{sec:motivation});
%
then, we introduce and discuss {\aser}, defining the main concepts,
syntax and semantics --- first informally (Section~\ref{sec:proposal})
and then providing the formalisation of some key aspect
(Section~\ref{sec:formalisation}).
%
We discuss then the results of a first evaluation that has been
carried out, based on a prototype implementation extending the ASTRA
platform (Section~\ref{sec:evaluation}).
%
We conclude the paper discussing related work
(Section~\ref{sec:related}) and sketching future work
(Section~\ref{sec:conclusion}).


