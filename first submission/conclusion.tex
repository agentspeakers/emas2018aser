%% RHB: OK to change this title as below?
%\section{Conclusion and Future Work}
\section{Conclusion}
\label{sec:conclusion}

In this paper, we introduced \aser, a novel extension of the classical
{\asl} language. The language provides encapsulation for agent
goals, which clearly improves legibility and reusability of AgentSpeak
code. Furthermore, the new language improves some of the shortcomings
of AgentSpeak in regards to goal orientation and declarative goals by
ensuring that all reactive plans are also associated with general
goals, providing a ``goal condition'' which means goals can be still
active even though presently there is no action for the agent to take
towards that goal, and allowing external events (i.e. reactions to
changes in beliefs) to trigger various plans, for all the goals it
might be relevant.  The proposal was implemented and 
 %formalised the main changes required in the
%existing formal semantics of AgentSpeak and 
experimentally evaluated on top of the ASTRA platform.

As with any new programming language, there is much future work, some
in fact ongoing. We are currently refining the ASTRA implementation,
trying to make a few optimisations to improve the evaluation results we
reported in this paper. A Jason-based implementation is also under
way; comparison of the performances of the two implementations might
lead to insight that might improve the implementation of the platforms
themselves. 

More generally, full understanding and evaluation of a programming
language takes many years. We expect in the long term to use \aser\ in
the practical development of multi-agent systems, both for real-world
systems and also academic ones (e.g., for the multi-agent programming
contest~\cite{Albrecht18}). However, besides the actual programming
practice, we expect \aser\ to contribute to formal work as
well. Assessing how formal verification of \aser\ systems compares to
the original language is also planned as future work.
%% RHB: OK to mention the multi-agent contest?

% \begin{itemize}
% \item refining current ASTRA implementation, that will allow to refine the evaluation
% \item Jason extension implementing {\aser}, and compare it with the ASTRA one
% \item stress {\aser} in practice 
% \item think about the tools: how existing IDE and tools can be extended to provide functionalities exploiting the new features
% \item investigate how the extension either improve or not the possibility to formally analyse the correctness properties
% \item ...
% \end{itemize}

